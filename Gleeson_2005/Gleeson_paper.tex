\documentclass[a4paper]{article}

\usepackage{natbib}
\usepackage[dvips,margin=3cm]{geometry}
\usepackage{graphicx}
\usepackage{amssymb}
\usepackage{units}
\usepackage{url}
\usepackage{flafter}

\title{\textbf{Assessing the efficacy of a touch screen overlay as a selection
device for typical GUI targets}}
\author{M.\ GLEESON\dag, N.\ STANGER\thanks{Corresponding author. Email nstanger@infoscience.otago.ac.nz.} \dag\ and E.\ FERGUSON\ddag}
\date{\dag Department of Information Science, \\
	\ddag Department of Human Nutrition, \\
	University of Otago, Dunedin, New Zealand}


\begin{document}

\renewcommand{\baselinestretch}{2}

\maketitle

\begin{quotation}
	\noindent In this paper we investigate the efficacy of a touch
	screen overlay compared to a mouse, when selecting typical graphical
	user interface (GUI) items in a desktop information system. A series
	of tests were completed involving multi-directional point and select
	tasks, and the results for both devices compared. The results showed
	that the touch screen overlay was not suitable for selecting GUI
	targets smaller than \unit[4]{mm}. The touch screen overlay was
	slower and had a higher error rate than the mouse, but there was no
	significant difference in throughput. Testers rated the mouse easier
	to use and to make accurate selections, while the touch screen
	overlay resulted in greater arm, wrist and finger fatigue. These
	results suggest that a touch screen overlay is not a practical
	selection device for desktop interfaces with small GUI targets.
\end{quotation}

\begin{quotation}
	\noindent \emph{Keywords:} Touch screen overlay; Mouse; Selection
	device; Fitts' Law; Performance evaluation; GUI item
\end{quotation}

% main text
\newcommand{\ISOnine}{ISO 9241-9}


\section{Introduction}
\label{sec-introduction}

Most modern information systems that run on desktop personal computers
are designed to be used with a keyboard and mouse. While the combination
of keyboard and mouse is the accepted method of interaction with such
systems it does not necessarily suit all information systems.
Information systems with limited data entry may be more usable through
the use of a keyboard and touch screen. Touch screens require less
physical space and thus the workstation environment in an office setting
could be improved, allocating more space to the user and less to the
computer.

The purpose of this paper is to investigate how effective a touch screen
overlay is compared to a mouse, when selecting typical graphical user
interface (GUI) items. The target types tested were buttons, check
boxes, combo boxes and text boxes, which are typical of those found in
an interface for an information system. Each target type was tested at
three different sizes (see Section~\ref{sec-GUI}).

A typical touch screen device comprises a monitor enhanced with hardware
for detecting touches on the screen surface. An alternative approach is
to attach a discrete touch-sensitive surface to an existing conventional
monitor. \citet{Sear-A-1991-IJMMS} have previously assessed the efficacy
of specialised touch screen hardware, but there appears to have been
little research into the efficacy of touch screen overlays. We therefore
chose to compare the performance of a touch screen overlay with that of
a mouse.

The testing occurred in the context of a research project undertaken by
the Department of Human Nutrition at the University of Otago. This
project aims to improve complementary feeding diets for toddlers in New
Zealand, by designing a computer program to help formulate
population-specific food-based dietary guidelines for this high risk
group. The program, which is based on a previously published linear
programming approach \citep{Ferg-EL-2004-Nutrition}, is a
decision-making tool, designed specifically for nutrition programme
planners to assist them in selecting appropriate and improved home-based
complementary foods.

% do we need this?
The remainder of the paper discusses our experiment and the results
obtained. Section~\ref{sec-GUI} briefly describes the types of targets
used in the experiment, while Section~\ref{sec-evaluation} describes the
measures used to evaluate the selection devices. The experimental design
is described in Section~\ref{sec-method}. Section~\ref{sec-analysis}
describes how the data were analysed, and Section~\ref{sec-results}
presents the results of the experiment. Our conclusions are presented in
Section~\ref{sec-conclusions}.


\section{GUI targets}
\label{sec-GUI}

Since the 1980's much work has gone into developing human computer
interface guidelines. Today's interfaces are made up of a combination of
different targets that include text boxes, check boxes, combo boxes,
list boxes, buttons, labels, tool bars, etc. \citet{Sear-A-1991-IJMMS}
showed that touch screens can be successfully used as a selection device
and can have advantages over a mouse, even for small targets. These
results were, however, based on selecting arbitrary shapes and not the
typical targets found in modern GUIs.

To accurately test the performance of each selection device within the
experiment, three different sizes of GUI target were used, corresponding
to small, medium and large GUI items. As our experiment took place in a
Windows environment, we would have preferred to base these sizes on
Microsoft's user interface guidelines. However, Microsoft's guidelines
specify only a single standard size for most GUI items \citep[pp.\
448--450]{MS-1999-UI}. \citeauthor{Appl-2004-HIG}'s
\citeyearpar{Appl-2004-HIG} human interface guidelines specify three
standard sizes (mini, small and large), but these proved to be rather
small in our Windows-based testing environment. We therefore adjusted
Apple's three sizes such that the ``small'' size was consistent with
Microsoft's guidelines. The resultant target sizes are listed in
Table~\ref{tab-target-sizes}. A screen resolution of \unit[81]{DPI} was
assumed.


\begin{table}[ht]
	\caption{Target sizes (width \(\times\) height) used in the experiment.}
	\label{tab-target-sizes}
	\begin{tabular}{llll}
		\hline
		\textbf{Target type}	&	\textbf{Large}							&	\textbf{Medium}							&	\textbf{Small}	\\
		\hline
		Text box				&	\(\unit[63]{mm} \times \unit[11]{mm}\)	&	\(\unit[55]{mm} \times \unit[8]{mm}\)	&	\(\unit[47]{mm} \times \unit[6]{mm}\)	\\
		Combo box				&	\(\unit[63]{mm} \times \unit[11]{mm}\)	&	\(\unit[55]{mm} \times \unit[8]{mm}\)	&	\(\unit[47]{mm} \times \unit[6]{mm}\)	\\
		Button					&	\(\unit[28]{mm} \times \unit[13]{mm}\)	&	\(\unit[24]{mm} \times \unit[9]{mm}\)	&	\(\unit[17]{mm} \times \unit[6]{mm}\)	\\
		Check box\(^{a}\)		&	\(\unit[9]{mm} \times \unit[9]{mm}\)	&	\(\unit[6]{mm} \times \unit[6]{mm}\)	&	\(\unit[4]{mm} \times \unit[4]{mm}\)	\\
		\hline
	\end{tabular}
	
	{\footnotesize \(^{a}\)This refers to the size of the check box itself, not the associated text label.}
\end{table}


\section{Evaluation methods}
\label{sec-evaluation}

Each selection device was assessed using a combination of performance
and comfort measures. The performance measures were primarily taken from
the \ISOnine\ standard \citep{ISO-2000-9241-9}, while the comfort
measures were derived from a questionnaire administered to test
participants.


\subsection{Performance}
\label{sec-evaluation-performance}

ISO 9241 specifies standards for the ergonomic design of office
computing environments. Part 9 of this standard describes different
tests that can be used to evaluate one or more pointing devices
\citep{ISO-2000-9241-9}. The standard describes a serial point and
select task and recognises a dependent measure used with this test,
known as \emph{throughput}. The serial test comprises moving the cursor
back and forth between two targets using the pointing device and
selecting each target by pressing and releasing a button on the pointing
device. One disadvantage of this approach is that only two targets are
used in the test and therefore interactions between more than two
targets---which often occur in a typical information system
interface---are not studied.

\citet{Mack-IS-2001-EHCI} note that throughput is a very important
measure, as it reflects the efficiency of the user completing the task
and is a measure of both speed and accuracy. Throughput is calculated by
the formula \(\mathit{throughput} = \mathit{ID}_{e} / \mathit{MT}\),
where \(\mathit{MT}\) is the movement time in seconds (defined as the
time taken to successfully select a target) and \(\mathit{ID}_{e}\) is
Fitts' \citeyearpar{Fitt-PM-1954-Law} \emph{index of difficulty}
measured in bits. Throughput is thus measured in bits per second (bps).

The index of difficulty is calculated by the formula \(\mathit{ID}_{e} =
\log_{2}((D / W_{e}) + 1)\), where \(D\) is the distance to the target
and \(W_{e}\) is the \emph{effective width} of the target. The effective
width reflects spatial variability in a sequence of trials, and thus
differs from the actual width of the target. The effective width of a
target is calculated by the formula \(W_{e} = 4.133 \times
\mathit{SD}_{x}\), where \(\mathit{SD}_{x}\) is the standard deviation
in the selection coordinates measured along the path to target \(x\).

\ISOnine\ does not provide any guidance on the range of index of
difficulty values to use in testing. \citet{Doug-SA-1999-CHI} recommend
using a range from 2 to 6 bits. They also recommend calculating the
\emph{error rate} as a separate dependent measure of accuracy. The
error rate is defined as the ratio of incorrect to correct selections
made on a target, so an error rate of 100\% implies that there were as
many errors made as correct selections. Error rate is not included in
\ISOnine, but has been used in several other studies
\citep{Sear-A-1991-IJMMS,Sear-A-1993-BIT,Hara-H-1996,Bend-G-1999-PhD,
Doug-SA-1999-CHI,Mack-IS-2001-EHCI,Po-BA-2004-CHI}. Computing both
throughput and error rate gives a more detailed performance analysis for
the selection device in question.


\subsection{Comfort}
\label{sec-evaluation-comfort}

ISO 9241-9 argues that to fully evaluate a selection device requires
assessment of user effort and comfort in addition to performance
measurements. Comfort is subjective and can be assessed by means of
questionnaires, while effort can be evaluated objectively by measuring
the biomechanical load on users as they use a device. Unfortunately,
such measurements require reasonably sophisticated equipment
\citep{Doug-SA-1999-CHI} that was not available to us. We therefore
omitted effort measurements from our experiment.

A questionnaire was used to assess comfort and user satisfaction for
each selection device in our experiment. The selection device assessment
questionnaire comprised sixteen questions, eight of which were taken
from the ISO ``Independent Questionnaire for Assessment of Comfort''
\citep{Doug-SA-1999-CHI}. The remaining eight questions related
specifically to the target types and target sizes that were tested. In
particular, the questionnaire aimed to assess the participants' comfort
in using the selection device, the difficulty in accurately selecting each
of the target types and the preferred size of each target type using the
selection device.

The responses to twelve of the questions were based on a five point
ordinal scale. The remaining four questions referred to the
participant's preferred size for each target type and were based on a
three point response corresponding to the target sizes tested---small,
medium and large (see Table~\ref{tab-target-sizes}). There was also a
space for participants to provide additional general feedback about the
testing process.


\subsection{Other considerations}
\label{sec-evaluation-other}

\citet{Doug-SA-1999-CHI} also note that \ISOnine\ does not take into
account any possible effects of learning, which can affect movement time
and accuracy. For example, \citet{Mack-IS-1991} found that the movement
times from the first of five testing sessions were significantly higher
than in later sessions. This can be explained as a result of learning
and shows that input device studies should take learning into account
and test for it; indeed, \citet{Doug-SA-1999-CHI} recommend applying a
repeated measures paradigm and testing for learning effects.

One interesting aspect of using typical GUI items as targets is the
variation in selection behaviour for different target types, compared to
earlier studies that used simple rectangular targets. The button, check
box and text box target types can be said to exhibit a ``one-step''
selection behaviour, because they require only single action (i.e., the
user clicks on them) in order to be selected. A combo box is different,
however, because it exhibits a ``two-step'' selection behaviour: first
the combo box must be selected in order to show the list of items, and
then an item must be selected from the displayed list, as illustrated in
Figure~\ref{fig-combo-box}. To complicate matters further, users may
execute this two-step behaviour using either a ``one-click'' or a
``two-click'' approach. In the former approach, the user clicks on the
combo box, drags down to the desired list item, then releases. In the
latter approach, the user clicks once on the combo box, then clicks
again on the desired list item. If the drop-down list were longer than
what could be displayed on screen, this could even lead to a
``multiple-click'' approach, where the user clicks multiple times on the
downward scroll arrow in the list. We have, however, not considered this
possibility in our experiment.


\begin{figure}[ht]
	\centering
	\includegraphics[scale=0.8]{combobox-step1}\quad\quad\quad
	\includegraphics[scale=0.8]{combobox-step2}
	\caption{The ``two-step'' action required to select a combo box.}
	\label{fig-combo-box}
\end{figure}



\section{Method}
\label{sec-method}

An experiment was carried out to test the effect of size for different
GUI target types with different selection devices. The experiment
involved participants completing a series of simple point and select
tasks. Small, medium and large sizes (see Table~\ref{tab-target-sizes})
were tested for combo boxes, text boxes, check boxes and buttons, using
either a touch screen overlay or a mouse. The test was
multi-directional, meaning the targets appeared in multiple directions
from the initial starting point. A variety of different sizes, angles
and distances were used for each target position.

The test itself comprised a screen containing a button in the centre and
a target for the participant to select as illustrated in
Figure~\ref{fig-test-environment}. When a participant clicked on the
centre ``Go'' button, a trial was started and a target appeared on the
screen. The trial ended when the participant successfully clicked the
target, which then disappeared. The time taken between clicking the
``Go'' button and successfully clicking on the target was recorded as
well as the number of errors made during the trial. The final
coordinates of the successful click on the target were recorded in order
to calculate the effective width of the target.


\begin{figure}
	\centering
 	\includegraphics[scale=0.5]{test-environment}
	\caption{Screenshot of the test environment with a target in the
	top left of the screen and the ``Go'' button in the centre.}
	\label{fig-test-environment}
\end{figure}


\subsection{Participants}
\label{sec-method-participants}

A participant sample size of twenty-four was used for the experiment.
Each participant was allocated to one of two groups with each group
using one selection device in testing. The allocation of groups was
based on the results of a questionnaire completed by each participant
prior to testing. The purpose of the pre-test questionnaire was to
establish the level of computer, mouse and touch screen experience of
each participant. Participants were then allocated to a selection device
group based on which device they had the least experience with.

Due to the testing being done within the nutrition program environment
mentioned in Section~\ref{sec-introduction}, the participants were all
nutritionists (i.e., typical users of the program). There were
twenty-one female and three male participants, all with a university
level of education. All participants were unpaid volunteers.


\subsection{Apparatus}
\label{sec-method-apparatus}

The test environment was implemented in Visual Basic.NET using Microsoft
Studio 2003, and is illustrated in Figure~\ref{fig-test-environment}.
During each test, data corresponding to the relevant measures (movement
time, number of errors and selection coordinates) were captured by the
software and automatically written to a Microsoft Excel worksheet.

The touch screen used in testing was a 17'' Magic Touch USB overlay
Model KTMT-1700-USB-M \citep{Keyt-2005-Touch}. This device uses a
take-off touch strategy, that is, a selection is not confirmed until the
user's finger is removed from the screen. An important property of touch
screen overlays is that they are placed over a conventional monitor and
the touch surface is thus not coincident with the display surface. This
can cause a slight discrepancy or parallax effect between where the user
touches the overlay and where the cursor is positioned on the screen.

The touch screen overlay was fitted to a Dell 15'' Flat Panel Model
E151FPb monitor. A flat panel monitor was chosen because it was noticed
during pre-testing that typical CRT monitors with curved screens
produced a variable gap between the overlay and the display surface,
thus potentially leading to a greater parallax effect than with a flat
display surface.

The mouse used in testing was a standard Dell PS/2 Optical Mouse Model
M071KC. Both devices were connected to a Dell Inspiron 7500 laptop
computer that ran the testing software.


\subsection{Design}
\label{sec-method-design}

A mixed design experiment was used with the selection device as a
between-subjects factor. The independent (between-subject) variables
were:
\begin{itemize}

	\item Target type (text box, combo box, button and check box)

	\item Target size (large, medium and small---see
	Table~\ref{tab-target-sizes})

	\item Target distance (\unit[40]{mm}, \unit[80]{mm} and
	\unit[160]{mm}---see below)

	\item Target angle (\(45^{\circ}\), \(135^{\circ}\), \(225^{\circ}\)
	and \(315^{\circ}\)---see below)

	\item Trial (1 to 144)

	\item Block (1 to 6)

\end{itemize}
The dependent variables within the experiment were throughput, movement
time and error rate.

The entire test was divided into six blocks. Each block contained every
possible combination of target type (four combinations), size (three
combinations), angle from initial starting point (four combinations) and
distance from initial starting point (three combinations). Consequently
there were 144 trials in each block and the entire experiment per
participant comprised a total of 864 trials (six blocks of 144 trials
each). Combinations of target type, distance and angle were presented to
the participant in random sequence with no repetition. Target size was
deliberately set to large for the first forty-eight trials in each block,
followed by medium for the next forty-eight trials, and finally small for
the remaining trials, in order to compensate for learning effects.

The combination of distance and angle from the initial starting point
yielded twelve possible target positions for each trial, as illustrated
in Figure~\ref{fig-target-positions}. Three distances were used that
represented target positions ranging from close to the initial starting
point to very far away from the initial starting point. Four angles were
chosen so that targets could be tested in ninety degree blocks and to
provide a good range of screen positions for the target. The first angle
was set to \(45^{\circ}\) with \(90^{\circ}\) increments thereafter, in
order to mimic real life user interface target selection, where targets
are situated in different areas of the screen and therefore selections
are made in multiple directions that are neither simply horizontal nor
vertical.


\begin{figure}
	\centering
	\includegraphics{target-positions}
	\caption{Positions of targets tested. The black box represents the
	initial starting point and the rounded rectangles represent the
	target positions.}
	\label{fig-target-positions}
\end{figure}


The index of difficulty (\(\mathit{ID}_{e}\)) was ascertained for each
possible task using the combination of distance and non-adjusted target
width. This showed that the test had a range of \(\mathit{ID}_{e}\)
values from \unit[0.7]{bits} (\unit[63]{mm} width and \unit[160]{mm}
distance) to \unit[5.4]{bits} (\unit[4]{mm} width and \unit[40]{mm}
distance). It is important to note that the combo box distance values
were adjusted in these calculations to reflect the two-step selection
behaviour of this target type (as discussed in
Section~\ref{sec-evaluation-other}). That is, we need to consider not
just the distance from the initial starting point to the target, but
also the extra distance from the main combo box to the selected list
item. In our experiment, participants were told to always select the
third list item in combo boxes (as illustrated in
Figure~\ref{fig-combo-box}), so the adjusted distance for a combo box
was equal to the normal distance from the initial starting point to the
target, \emph{plus} the additional distance to the third list item.
Additional data about the selections made on combo boxes were recorded
in order to account for the selection approach of the participant,
whether it be a ``one-click'' or ``two-click'' approach.


\subsection{Procedure}
\label{sec-method-procedure}

The participant was given an introduction to the test by the research
observer, which included a brief summary of the aims of the study and
what the test involved. The participant was also given an instruction
sheet that they had access to throughout the duration of the test. After
reading the instruction sheet the participant had the opportunity to ask
questions or raise any issues.

Participants were instructed to complete each block of trials as quickly
as possible without losing accuracy. Participants were given the
opportunity to rest for as long as they wished between blocks. It was
made clear to participants that a task was only complete once the target
was successfully selected.

Before the test began, participants were instructed to complete a
practice session involving fifteen random trials of the same point and
select tasks used in the test. This brought all participants up to a
minimal level of experience with their selection device. This also meant
that each participant knew how to correctly select each target type
including the combo box.

At the conclusion of the test the participant was asked to fill out a
questionnaire regarding comfort and user satisfaction with the selection
device used.


\section{Analysis}
\label{sec-analysis}

The data collected from the software included movement time, error rate
and throughput and was used to evaluate selection device performance. A
mixed design repeated measures analysis of variance model (MANOVA) was
used for movement time and throughput to examine within subject
differences in target type and size, as well as between subject
differences in selection device. A Greenhouse and Geisser correction of
the F-ratio was used whenever the Mauchly's test results showed that
assumptions of sphericity were violated.

Post hoc tests, for multiple comparisons, were made using the Bonferroni
method. Due to the skew observed in the error rate data (see
Section~\ref{sec-results-learning}), inter-device differences in error
rates were assessed using the Mann-Whitney U Test.

The comfort questionnaire was based on a five point ordinal scale, where
five generally indicated a poor rating. Because of the small data size, a
Mann-Whitney (non-parametric) test was used.

All statistical analyses were performed using SPSS version 11.0. A
\emph{p}-value of \(< 0.05\) was considered statistically significant.


\section{Results and discussion}
\label{sec-results}


\subsection{Adjusting for learning}
\label{sec-results-learning}

\citet{Doug-SA-1999-CHI} have shown that the effects of learning can
affect movement time and accuracy. They therefore recommend that input
device studies should apply a repeated measures paradigm and test for
learning effects.

From analysing the results of movement time and throughput over each
test block, it was clear for the combo box and check box target types
that learning occurred from the first to the second block with the touch
screen overlay, as seen in Figure~\ref{fig-movement-time-learning}. Due
to prior participant experience, no learning was observed with the
mouse. No learning occurred with the text box or button, most likely due
to their relatively large size and simple selection behaviour.


\begin{figure}
	\centering
	\includegraphics{combobox-learning}
	\includegraphics{checkbox-learning}
	\caption{Learning is displayed for movement time by selection device
	and test block for the combo box and check box.}
	\label{fig-movement-time-learning}
\end{figure}


Statistical analysis using a simple repeated measure ANOVA was carried
out on movement time for both the check box and combo box. For movement
time of the combo box, the effect of block \(\times\) device was
significant \((F(1.549, 1335.219) = 4.373, p < 0.05)\). Helmhert
contrasts showed that the differences between blocks became
non-significant after block 1 \((p > 0.05)\), which implies that
learning occurred during block 1.

For movement time of the check box, the effect of block \(\times\)
device was significant \((F(1.608, 1385.960) = 4.763, p < 0.05)\).
Helmhert contrasts again showed that the differences between blocks
became non-significant after block 1 \((p > 0.05)\), implying that
learning occurred during block 1.

To account for learning with the combo and check box, results from block
6 only were used to calculate the movement time and throughput measures,
as the results from block 6 alone gave a good measure of performance.
However, the error rate results were highly skewed for both target
types, with two participants accounting for almost 90\% of the errors.
Error rates for the check box and combo box were therefore calculated
using results from all six test blocks.


\subsection{Movement time}
\label{sec-results-movement}

The results showed that the mouse had an overall movement time of
\unit[1.32]{s} across all target types compared to \unit[1.57]{s} for
the touch screen overlay. We can thus conclude that the touch screen
overlay was on average 1.2 times slower than the mouse. This is
interesting because \citet{Sear-A-1991-IJMMS} found that the movement
times for mouse and touch screen monitor were similar for rectangular
targets larger than \unit[2]{mm}. Therefore the nature of the two types
of touch screen (overlay versus monitor) may affect the movement time
associated with the type of touch screen. It is also likely that due to
the loss of accuracy found with the overlay during testing, the touch
screen monitor will have faster movement times compared to the touch
screen overlay.

The movement times for each target type showed that the text box had the
fastest movement time, followed by the button, the check box and the
combo box. These results are illustrated in
Figure~\ref{fig-movement-time} and are consistent with Fitts' Law
\citep{Fitt-PM-1954-Law}, in that the largest target (the text box) had
the fastest movement time. It may appear that the combo box results
violate Fitts' Law, as the combo box is the same size as the text box,
yet is over twice as slow. In this case however, the slow movement times
are not a function of the target size, but rather a result of the more
complex two-step behaviour required to successfully select a combo box
(which is not considered by Fitts' Law). The extra movement of selecting
an item from the drop-down list clearly dramatically increases the
movement time for the combo box. As the additional distance from the
main combo box area to the list item is relatively short, the
significant increase in movement time is therefore most likely due to
users making more errors.


\begin{figure}
	\centering
	\includegraphics{movement-time}
	\caption{Movement time by target type and device, averaged across
	all target sizes.}
	\label{fig-movement-time}
\end{figure}


The touch screen overlay was found to have similar movement time to the
mouse for the medium and large targets, but for the small targets, the
touch screen overlay was on average 1.5 times slower than the mouse. The
only time that the touch screen overlay was found to be faster than the
mouse was with the largest target type (the large text box), but the
difference was less than one percent.

The movement time for the small check box on the touch screen overlay
was about 2.7 times slower than that for the mouse. The small check box
was the smallest item tested, with dimensions of \unit[4]{mm} \(\times\)
\unit[4]{mm}. We can conclude that the touch screen overlay was not
efficient for selecting targets as small as \unit[4]{mm}. Compare this
with \citet{Sear-A-1991-IJMMS}, who showed that a touch screen monitor
has similar movement time to a mouse for targets as small as
\unit[2]{mm}. While a touch screen monitor can be used with targets as
small as \unit[2]{mm}, a touch screen overlay should only be used for
targets with a size of greater than \unit[4]{mm}. The results from the
error rate analysis also support this conclusion (see
Section~\ref{sec-results-error-rate}).


\subsection{Throughput}
\label{sec-results-throughput}

The average throughput for the mouse was \unit[1.242]{bps}, slightly
higher than the \unit[1.214]{bps} average throughput for the touch
screen overlay. The selection device by itself was shown not to have a
significant effect on throughput \((F(1, 22) = 0.02, p > 0.05)\).
Throughput did not vary for size but throughput did vary depending on
target type \((F(2.07, 45.55) = 4.77, p < 0.001)\), as shown in
Figure~\ref{fig-throughput}. Check boxes had the highest throughput rate
of \unit[1.967]{bps} (sd = 0.720). This is interesting as the check box
had the second worst movement time and the worst error rate. Upon
further investigation it was seen that the movement time for the check
box was in fact in the middle range of all targets, and due to its small
size it had a high index of difficulty. These two factors are the most
likely reason for the check box having such a high throughput rate.


\begin{figure}[ht]
	\centering
	\includegraphics{throughput}
	\caption{Throughput by target type and device, averaged across all
	target sizes.}
	\label{fig-throughput}
\end{figure}


The combo box had the worst average throughput of \unit[0.501]{bps} (sd
= 0.213). The index of difficulty was not very high for the combo box,
so its low throughput rate could be attributed to its high movement
time.

The average throughput rate of \unit[1.242]{bps} for the mouse is much
lower than that found by previous research. A study by
\citet{Doug-SA-1994-SIGCHI} showed that a mouse had a throughput rate of
\unit[4.15]{bps}. \citet{Mack-IS-1991} compared three devices (mouse,
tablet and trackball) using four target sizes (8, 16, 32 and 64 pixels)
over two different types of tasks (pointing and dragging), and found
that the throughput for the mouse was \unit[4.5]{bps}. This may indicate
that the level of selection difficulty in our experiment is higher than
in previous research. This could be due to the selection of GUI targets
instead of arbitrary rectangular targets.


\subsection{Error rate}
\label{sec-results-error-rate}

The average error rate for the mouse was only 2.7\% which is consistent
with previous studies. The touch screen overlay, on the other hand, had
an average error rate of 60.8\%. \citet{Sear-A-1991-IJMMS} found that a
touch screen monitor had an average error rate of 49\% but this was
across much smaller targets. This suggests that there is a loss of
accuracy when using a touch screen overlay as opposed to a touch screen
monitor.

The check box had a significantly high average error rate of 78.5\%
across all sizes and both devices; in particular, the small check box on
the touch screen overlay had an error rate of 393\%. The touch screen
overlay incurred the majority of the errors. For all sizes of the check
box the mouse produced an average error rate of 6.7\%, while the touch
screen overlay produced an average error rate of 178.5\%. The
distinguishing factor of the check box compared to the other target
types is its small size, so we can conclude from this that a touch
screen overlay is more inaccurate for selecting small targets
(\unit[4]{mm} or less).

The button and text box had much lower error rates than the check box
and combo box (as seen in Figure~\ref{fig-error-rate}). As the button
and text box also exhibited low movement times, we can conclude that
these two target types have very good overall performance.


\begin{figure}
	\centering
	\includegraphics{error-rate}
	\caption{Error rate by target type and device, averaged across all
	target sizes.}
	\label{fig-error-rate}
\end{figure}


Note that the error rate calculation for combo boxes assumes a
``two-click'' selection approach, as only two of the twenty-four
participants used the ``one-click'' approach. Both of these participants
used the mouse.


\subsection{Comfort}
\label{sec-results-comfort}

In terms of accurate pointing the mouse (2.083) was rated easier than
the touch screen overlay (3.000). These differences were statistically
significant \((p < 0.01)\). The responses regarding the question on
neck, wrist and arm fatigue showed that the touch screen overlay had a
high rating (4.083), whereas the mouse was rated in the midpoint range
(3.167). These differences were statistically significant \((p < 0.05)\).
The final question rated the overall difficulty in using the selection
device. The mouse (4.250) was rated easier to use than the touch screen
overlay (3.333). These differences were statistically significant \((p <
0.05)\).

Participants using the touch screen overlay rated both the text box and
button as easy to accurately select, with the large and medium sizes
being most preferred. This feedback is consistent with the data
collected in that the text box and button have low movement times and
low error rates (i.e., they are easy to accurately select). The combo
box was rated in the middle of the range, and the check box was rated as
very hard to select. Three quarters of the touch screen overlay
participants preferred to select large combo boxes and check boxes,
which is consistent with the high error rates and movement times
associated with these two target types on the touch screen overlay.

Participants using the mouse also rated the text box and button as easy
to accurately select, with the large and medium sizes being most
preferred. Both the combo box and check box were rated harder to select
than the button and text box with the check box having the worst rating.
As with the touch screen overlay, participants preferred large combo
boxes and check boxes.

In the general feedback, one participant noted the lack of arm support
when selecting targets at the top of the screen. This is an interesting
comment, because the nature of using a touch screen means the user's arm
might be raised off the desk, and thus be self-supporting when selecting
items towards the top of the screen.

Another suggestion was making the target change colour when the cursor
is located above it. This is a similar concept to that of interactive
rollover items commonly used in web pages. Auditory feedback has been
shown to affect speed and accuracy when making a selection
\citep{Bend-G-1999-PhD}, so it is likely that the visual feedback
received from GUI targets will affect the selection performance. All of
the target types tested provide some form of immediate visual feedback,
from the button being visually depressed to a tick appearing in the
check box. Further study is needed to assess how visual feedback affects
selection performance of GUI items and what is the most effective method of
providing feedback.


\subsection{Other findings}
\label{sec-results-other}

Two interesting patterns emerged when we calculated the standard
deviation of the final selection coordinates on the screen. First, when
selecting text boxes with the mouse, there was greater variation in
final selection coordinates on the right side of the screen
(\(45^{\circ}\) and \(315^{\circ}\) target angles) than on the left side
of the screen (\(90^{\circ}\) and \(225^{\circ}\) target angles). This
behaviour was not apparent when selecting text boxes with the touch
screen overlay, as illustrated in Figure~\ref{fig-variation-textbox}.
This would mean that participants using the mouse were more careful when
making selections with text boxes on the left side of the screen than on
the right.


\begin{figure}
	\centering
	\includegraphics{variation-text-mouse}
	\includegraphics{variation-text-touch}
	\caption{Standard deviation of final selection coordinates for the
	text box, averaged across all target sizes.}
	\label{fig-variation-textbox}
\end{figure}


Second, and even more interesting, was the observation that while
selections made on the combo box with the mouse exhibited similar
behaviour to the text box, the behaviour on the touch screen overlay was
the opposite (although less pronounced), as shown in
Figure~\ref{fig-variation-combobox}. That is, selections of combo boxes
on the left side of the touch screen overlay had greater variation than
selections made on the right.


\begin{figure}
	\centering
	\includegraphics{variation-combo-mouse}
	\includegraphics{variation-combo-touch}
	\caption{Standard deviation of final selection coordinates for the
	combo box, averaged across all target sizes.}
	\label{fig-variation-combobox}
\end{figure}


These effects were most pronounced for the large size of both the text
box and the combo box. The variation of selection coordinates for the
smaller targets (the button and check box) was consistent across both
the mouse and touch screen overlay and for all target sizes.

We can only speculate as to the reasons for this variation. In the case
of the combo box, one possibility is its asymmetrical appearance
compared to the other target types, which may encourage participants to
try to click specifically on the drop-down arrow of the combo box,
rather than treating the entire combo box as a target (this could also
be another factor in the slow movement times for this target type).
However, this does not explain the variation between left and right
sides of the screen, nor why the same behaviour was observed with the
completely symmetrical text box.

Another possibility is the handedness of the participants, which in the
case of a touch screen might affect how difficult it is to select
targets on different sides of the screen. Unfortunately we did not ask
participants whether they were right- or left-handed, and thus can draw
no conclusions on this point.


\section{Conclusion}
\label{sec-conclusions}

The goal of our study was to assess the efficacy of a touch screen
overlay compared to a mouse, when selecting the typical GUI targets
commonly presented to users in desktop information systems. This was
achieved by an experiment measuring movement time, throughput and error
rate for various combinations of target type, size, angle and distance.
Comfort and user satisfaction were assessed by means of a questionnaire.

The results showed that the touch screen overlay was both slower and
less accurate than the mouse. The touch screen overlay was found to have
reasonable performance with large GUI items but poor performance with
smaller GUI items. The touch screen overlay did have comparable movement
times to the mouse for medium and large sized targets. Throughput did
not vary across device or target size but did vary across target type.
Both selection devices had the same user preference except with respect
to the smallest target type (check boxes), in which the mouse had a
higher preference. The mouse was rated easier to make accurate
selections with than the touch screen overlay. The touch screen overlay
also had worse arm, wrist and finger fatigue compared to the mouse. From
these results we can conclude that the mouse had higher user
satisfaction than a touch screen.

An unusual variation in final selection coordinates was noted for both
text boxes and combo boxes. Further studies are required to establish
why this variation occurs.

In general we can conclude that a touch screen overlay with no external
device (e.g., a pen) is not an effective selection device for targets
with dimensions of \unit[4]{mm} or smaller. When designing interfaces
that will be used with a touch screen overlay, selection within the
interface will be more efficient if the GUI items are larger than
\unit[4]{mm}.

Although the results showed that the touch screen overlay was not
efficient and usable for selecting GUI items with a size of \unit[4]{mm}
or less, this may not be the case when a pen or some external device is
used in conjunction with the touch screen overlay. In general there
seems to be a lack of research in device assessment of touch screen
overlays with pens or other external devices. Further testing on touch
screen overlays used with an external device such as a pen may well show
that a touch screen overlay is adequate and efficient for selecting
small GUI items.


% The Appendices part is started with the command \appendix;
% appendix sections are then done as normal sections
% \appendix

% \section{}
% \label{}

% Bibliographic references with the natbib package:
% Parenthetical: \citep{Bai92} produces (Bailyn 1992).
% Textual: \citet{Bai95} produces Bailyn et al. (1995).
% An affix and part of a reference:
%   \citep[e.g.][Ch. 2]{Bar76}
%   produces (e.g. Barnes et al. 1976, Ch. 2).

\bibliographystyle{elsart-harv}
\bibliography{Gleeson_paper}
% \begin{thebibliography}{}

% \bibitem[Names(Year)]{label} or \bibitem[Names(Year)Long names]{label}.
% (\harvarditem{Name}{Year}{label} is also supported.)
% Text of bibliographic item

% \bibitem[]{}
% 
% \end{thebibliography}

\end{document}

