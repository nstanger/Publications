% Template article for preprint document class `elsart'
% with harvard style bibliographic references
% SP 2001/01/05

\documentclass{elsart}

% Use the option doublespacing or reviewcopy to obtain double line spacing
% \documentclass[doublespacing]{elsart}

% the natbib package allows both number and author-year (Harvard)
% style referencing;
\usepackage[authoryear,sort]{natbib}

% if you use PostScript figures in your article
% use the graphics package for simple commands
% \usepackage{graphics}
% or use the graphicx package for more complicated commands
\usepackage{graphicx}
% or use the epsfig package if you prefer to use the old commands
% \usepackage{epsfig}

% The amssymb package provides various useful mathematical symbols
\usepackage{amssymb}

\usepackage{units}
\usepackage{url}


\begin{document}

\begin{frontmatter}

% Title, authors and addresses

% use the thanksref command within \title, \author or \address for footnotes;
% use the corauthref command within \author for corresponding author footnotes;
% use the ead command for the email address,
% and the form \ead[url] for the home page:
% \title{Title\thanksref{label1}}
% \thanks[label1]{}
% \author{Name\corauthref{cor1}\thanksref{label2}}
% \ead{email address}
% \ead[url]{home page}
% \thanks[label2]{}
% \corauth[cor1]{}
% \address{Address\thanksref{label3}}
% \thanks[label3]{}

\title{Assessing the efficacy of a touch screen overlay as a selection
device using typical GUI targets}

% use optional labels to link authors explicitly to addresses:
% \author[label1,label2]{}
% \address[label1]{}
% \address[label2]{}

\author[info]{Matthew Gleeson},
\ead{matt\_gleeson@xtra.co.nz}
\author[info]{Nigel Stanger\corauthref{cor}},
\ead{nstanger@infoscience.otago.ac.nz}
\corauth[cor]{Corresponding author. Tel.: +64-3-479-8179; fax: +64-3-479-8311}
\author[hunt]{Elaine Ferguson}
\ead{elaine.ferguson@stonebow.otago.ac.nz}

\address[info]{Department of Information Science,}
\address[hunt]{Department of Human Nutrition, \\ University of Otago, PO Box 56, Dunedin, New Zealand}


\begin{abstract}

Touch screens are a popular method of interacting with kiosk-based
information systems, whereas typical desktop information systems use a
mouse as the primary selection device. In this paper we investigate the
efficacy of a touch screen overlay compared to a mouse, when selecting
typical graphical user interface (GUI) items in a desktop information
system. A series of tests were completed involving multi-directional
point and select tasks, and the results for both devices compared. The
results showed that the touch screen overlay was not suitable for
selecting GUI targets smaller than \unit[4]{mm}. The touch screen
overlay was slower and had a higher error rate than the mouse, but there
was no significant difference in throughput. Testers rated the mouse
easier to use and to make accurate selections, while the touch screen
overlay resulted in greater arm, wrist and finger fatigue. These results
suggest that a touch screen overlay is not a practical selection device
for desktop interfaces with small GUI targets.

\end{abstract}

\begin{keyword}
Touch screen overlay \sep
Mouse \sep
Selection device \sep
Fitts' Law \sep
Performance evaluation \sep
GUI items
\end{keyword}

\end{frontmatter}

% main text
\section{Introduction}
\label{sec-introduction}

Most modern information systems that run on desktop personal computers
are designed to be used with a keyboard and mouse. While the combination
of keyboard and mouse is the accepted method of interaction with such
systems it does not necessarily suit all information systems.
Information systems with limited data entry may be more usable through
the use of a keyboard and touch screen. Touch screens require less
physical space and thus the workstation environment in an office setting
could be improved, allocating more space to the user and less to the
computer.

The purpose of this paper is to investigate how effective a touch screen
overlay is compared to a mouse, when selecting typical graphical user
interface (GUI) items used in information systems. The target types
tested were buttons, check boxes, combo boxes and text boxes, which are
typical of those found in an interface for an information system. Each
target type was tested at three different sizes (see
Section~\ref{sec-GUI}).

A typical touch screen device comprises a monitor enhanced with hardware
for detecting touches on the screen surface. An alternative approach is
to attach a discrete touch-sensitive surface to an existing conventional
monitor. \citet{Sear-A-1991-IJMMS} have previously assessed the efficacy
of specialised touch screen hardware, but there appears to have been
little research into the efficacy of touch screen overlays. We therefore
chose to compare the performance of a touch screen overlay with that of
a mouse.

The testing occurred in the context of a research project undertaken by
the Department of Human Nutrition at the University of Otago. This
project aimed to improve complementary food diets for toddlers in New
Zealand by designing a computer program to help formulate
population-specific food-based dietary guidelines for this high risk
group. The program was a rapid assessment decision-making tool, designed
specifically to assist nutrition programme planners in selecting
appropriate and improved home-based complementary foods
\citep{Ferg-E-2004}. The testing for our study took place within this
program environment.

% do we need this?
The remainder of the paper discusses the experimental design and the
results obtained. Section~\ref{sec-GUI} briefly describes the types of
targets used in the experiment, while Section~\ref{sec-evaluation}
describes the measures used to evaluate the selection devices. The
experimental design is described in Section~\ref{sec-method}.
Section~\ref{sec-analysis} describes how the data were analysed, and
Section~\ref{sec-results} presents the results of the experiment. Our
conclusions are presented in Section~\ref{sec-conclusions}.


\section{GUI targets}
\label{sec-GUI}

Since the 1980s much research has been put into developing human
computer interface guidelines. Today's interfaces are made up of a
combination of different targets that could include text boxes, check
boxes, combo boxes, list boxes, buttons, labels, tool bars etc. Results
of research undertaken by \citet{Sear-A-1991-IJMMS} showed that touch
screens can be successfully used as a selection device and can have
advantages over a mouse, even for small targets. But these results were
based on selecting arbitrary shapes and not typical GUI items commonly
found in today's GUIs.

To accurately test the performance of each selection device within the
experiment, varying sized targets were used. The sizes represent small,
medium and large for each target.

Apple has produced some guidelines for developers to follow when
designing interfaces for software \citep{Appl-2004-HIG}. In particular,
for each target three different sizes are given corresponding to large,
small and mini. The sizes used in this experiment were based on these
guidelines and are shown in Table~\ref{tab-target-sizes}.


\begin{table}
	\caption{Target sizes (width \(\times\) height) used in the experiment.}
	\label{tab-target-sizes}
	\begin{tabular}{llll}
		\hline
		\textbf{Target type}	&	\textbf{Large}							&	\textbf{Medium}							&	\textbf{Small}	\\
		\hline
		Text box				&	\(\unit[63]{mm} \times \unit[11]{mm}\)	&	\(\unit[55]{mm} \times \unit[8]{mm}\)	&	\(\unit[47]{mm} \times \unit[6]{mm}\)	\\
		Combo box				&	\(\unit[63]{mm} \times \unit[11]{mm}\)	&	\(\unit[55]{mm} \times \unit[8]{mm}\)	&	\(\unit[47]{mm} \times \unit[6]{mm}\)	\\
		Button					&	\(\unit[28]{mm} \times \unit[13]{mm}\)	&	\(\unit[24]{mm} \times \unit[9]{mm}\)	&	\(\unit[17]{mm} \times \unit[6]{mm}\)	\\
		Check box				&	\(\unit[9]{mm} \times \unit[9]{mm}\)	&	\(\unit[6]{mm} \times \unit[6]{mm}\)	&	\(\unit[4]{mm} \times \unit[4]{mm}\)	\\
		\hline
	\end{tabular}
\end{table}


\section{Evaluation methods}
\label{sec-evaluation}

Each selection device was assessed using a combination of performance
and comfort measures.

\subsection{Performance}
\label{sec-evaluation-performance}

The dependent measure described by ISO 9241-9 is throughput
\citep{ISO-2000-9241-9}. Throughput is noted by
\citet{Mack-IS-2001-EHCI} as being a very important measure as it
reflects the efficiency of the user completing the task and it is a
measure of both speed and accuracy. The formula for calculating
throughput is given in the following equation:
\[
	\mathit{throughput} = \mathit{ID}_{e} / \mathit{MT}
\]
where
\[
	\mathit{ID}_{e} = \log_{2}(D / W_{e} + 1)
\]

\(\mathit{ID}_{e}\) represents the index of difficulty and is defined in
terms of bits whereas movement time is defined in terms of seconds.
Therefore throughput is measured in bits per second (bps). \(D\)
represents the distance to the target. Throughput is used by the ISO
9241-9 standard as the performance measurement.

\(W_{e}\) is the effective width and differs from the width of the
target. It reflects the spatial variability in the sequence of trials.
The formula for calculating effective width is given in the following
equation:
\[
	W_{e} = 4.133 \times \mathit{SD}_{x}
\]
	
\(\mathit{SD}_{x}\) is the standard deviation in the selection
coordinates measured along the path to the target

Due to the nature of making a selection within a drop-down list, the
throughput for a combo box was adjusted to take into account the extra
distance to the desired list item.

Movement time is defined as time taken to successfully select a target.
Error rate is defined as the number of selections made outside of the
intended target. The error rate is the ratio of incorrect selections to
correct selections made, so an error rate of 100\% means as many errors
as correct selections.

Both movement time and error rate were left out of the ISO standard
9241-9 in terms of performance measurements but \citet{Doug-SA-1999-CHI}
recommend computing both measurements to give a more detailed
performance analysis for the selection device.

\subsection{Comfort}
\label{sec-evaluation-comfort}

A questionnaire was used to assess comfort and user satisfaction of each
selection device. The selection device assessment questionnaire
consisted of sixteen questions, eight of which were taken from the ISO
``Independent Questionnaire for Assessment of Comfort''
\citep{Doug-SA-1999-CHI}. The remaining eight questions related
specifically to the targets tested and the size of the targets tested.

In particular, the questionnaire aimed to assess the participants'
comfort in using the input device, the difficulty in accurately
selecting each of the targets and the preferable size of each target
using the input device.

The responses to twelve of the questions were based on a five point
ordinal scale. The remaining four questions referred to the
participant's preferred size for each target and were based on a three
point response corresponding to the sizes tested---small, medium and
large.

\section{Method}
\label{sec-method}

An experiment was carried out to test the effect of size for different
GUI targets with different selection devices. The experiment consisted
of completing a series of simple point and select tasks. Small, medium
and large sizes were tested for a combo box, text box, check box and
button. The selection devices tested were a touch screen overlay and
mouse. The test was multi-directional, meaning the target appeared in
more than one direction to the user. A variety of different sizes,
angles and distances were used for each target position.

\subsection{Participants}
\label{sec-method-participants}

A participant sample size of twenty four was used for the experiment.
Each participant was allocated to one of two groups with each group
using one selection device in testing.

The allocation of groups was based upon the results of a questionnaire
completed by each participant prior to testing. The purpose of the
questionnaire was to establish the level of computer, mouse and touch
screen experience of each participant. A participant was allocated to a
selection device group depending on what device they had the least
amount of experience with.

Due to the testing being done within a nutrition program environment,
the participants were all nutritionists (typical users of the program).
There were 21 female and 3 male participants with all having a
university level of education. All participants were unpaid volunteers.

\subsection{Apparatus}
\label{sec-method-apparatus}

Software written in Visual Basic.Net with Microsoft Studio 2003 was used
to implement the test as illustrated in
Figure~\ref{fig-test-screenshot}. Each test was connected to a Microsoft
Excel worksheet and the data corresponding to the relevant measures
(movement time, number of errors and selection coordinates) were
captured using the software and written to the Excel worksheet.

\begin{figure}
	\includegraphics{test-screenshot}
	\caption{Screenshot of the test with the target in the top left of
	the screen and the ``go'' button in the centre.}
	\label{fig-test-screenshot}
\end{figure}

The touch screen used in testing was a 17'' Magic Touch USB overlay
Model KTMT-1700-USB-M. This touch screen uses a lift-off touch strategy.
A touch screen overlay is a piece of equipment external to the monitor.
It sits in front of the monitor and behaves similarly to a touch screen
monitor. Using an overlay results in a gap between the overlay and the
monitor itself, this causes a slight discrepancy between where the user
touches the overlay and where the cursor is positioned on the screen.

The touch screen overlay was fitted to a Dell 15'' Flat Panel Model
E151FPb monitor. A flat panel monitor was chosen because it was noticed
during pre-testing that typical CRT monitors with rounded screens caused
a gap between where the users touched the screen and where the cursor
gets positioned. A flat panel is less likely to suffer this problem. The
device used for testing the mouse was a Dell PS/2 Optical Mouse Model
M071KC. Both devices were connected to a Dell Inspiron 7500 laptop
computer which ran the testing software.

\subsection{Procedure}
\label{sec-method-procedure}

The participant was initially given an introduction to the test by the
research observer. The introduction included a brief summary of the aims
of the study and what the test involved. The participant was also given
and told to read an instruction sheet which they had access to
throughout the duration of the test. After reading the instruction sheet
the participant had the opportunity to ask questions or raise any
issues.

Participants were instructed to complete each block of tests as quickly
as possible without losing accuracy. In between blocks of tasks,
participants were given the opportunity to rest for as long as they
wished. It was made clear to the participant that a task was only
complete once the target was successfully selected. Selecting the
button, check box and text box required the participant to simply click
on the target. The strategy required to select a combo box was
different. A combo box is a two-step target compared to the other
targets which were simple one-step targets. First the combo box must be
selected in order to show the list of items and then an item from the
displayed list must be selected. During the testing the participant was
instructed to always select the third item in the list when selecting a
combo box as illustrated in Figure~\ref{fig-combo-box}.

\begin{figure}
	\caption{The two-step action required to select the combo box.}
	\label{fig-combo-box}
\end{figure}

The participant was then instructed to complete a practice task
involving fifteen random trials of the same point and select tasks used
in the test. This brought all participants to a minimal level of
experience with their selection device. This also meant each participant
knew how to correctly select each device including the combo box.

At the conclusion of the test the participant was required to fill out a
questionnaire regarding comfort and user satisfaction with the selection
device used.

\subsection{Design}
\label{sec-method-design}

A mixed design experiment was used with the selection device as a
between-subjects factor. The independent (between-subject) variables
were:

\begin{itemize}

	\item Target Type (text box, combo box, button and check box)

	\item Target Size (large, medium and small)

	\item Target Distance (\unit[40]{mm}, \unit[80]{mm} and
	\unit[160]{mm})

	\item Target Angle (\(45^{\circ}\), \(135^{\circ}\), \(225^{\circ}\)
	and \(315^{\circ}\))

	\item Trial (1 to 144)

	\item Block (1 to 6)

\end{itemize}

The entire test was divided into six blocks. Each block contained every
possible combination of target type (4), size (3), angle from starting
point (4) and distance (3). There were 144 trials in each block and the
entire experiment per participant consisted of a total of 864 trials
(six blocks of 144 trials).

The different combinations of target location on the screen are
illustrated in Figure~\ref{fig-target-positions} and are a combination
of distance and the angle from the starting point.

\begin{figure}
	\includegraphics[scale=0.85]{target-positions}
	\caption{Positions of targets tested. The black box represents the
	starting point and the rounded rectangles represent the target
	positions.}
	\label{fig-target-positions}
\end{figure}


The dependent variables within the experiment were throughput (TH),
movement time (MT) and error rate (ER).

The index of difficulty was ascertained for each task using the
combination of distance and width. This showed that the test had a range
of Fitts' Index of Difficulty values from 0.7 bits (\unit[63]{mm} width
and \unit[160]{mm} distance) to 5.4 bits (\unit[4]{mm} width and
\unit[40]{mm} distance). The ordering of the target size presented to
the participant within each block in the experiment was deliberately set
to large, medium and lastly small to compensate for learning.

\section{Analysis}
\label{sec-analysis}

The data collected from the software included movement time, error rate
and throughput and was used to evaluate selection device performance. A
mixed design repeated measures analysis of variance model (MANOVA) was
used for movement time and throughput to examine within subject
differences in target and size, as well as between subject differences
in device. A Greenhouse and Geisser correction of the F-ratio was used
whenever the Mauchly's test results showed that assumptions of
sphericity were violated.

Post hoc tests, for multiple comparisons, were made using the Bonferroni
method. Due to the skew observed in the error rate data inter-device
difference in error rates were assessed using the Mann-Whitney U Test.

The comfort questionnaire was based on a five point ordinal scale. In
general five indicated a bad rating. Because of the small data size, a
Mann-Whitney (non-parametric) test was used. All statistical analyses
were performed using SPSS version 11.0.

\section{Results and discussion}
\label{sec-results}

\subsection{Adjusting for learning}
\label{sec-results-learning}

\citet{Doug-SA-1999-CHI} recommend that input device studies should
apply a repeated measures paradigm and test for learning effects. The
effects of learning have been shown to affect movement time and accuracy
\citep{Doug-SA-1999-CHI}.

From analysing the results of movement time and throughput over each
test block, it is clear for the combo box and check box that learning
occurs from the first to second block with the touch screen (as seen in
Figure~\ref{fig-movement-time-learning}). Due to prior experience, no
learning is observed with the mouse. No learning occurs with the text
box or button most likely due to their large size and simple selection
behaviour.

\begin{figure}
	\caption{Learning is displayed for movement time by device and block
	for the combo box and check box.}
	\label{fig-movement-time-learning}
\end{figure}

Statistical analysis using a simple repeated measure ANOVA was carried
out on movement time for both the check box and combo box. For movement
time of the combo box, the effect of block * device was significant
\((F(1.549, 1335.219) = 4.373, p < 0.05)\).

Helmhert contrasts show that the differences between blocks become
non-significant after block 1 \((p > 0.05)\). For movement time of the
check box, the effect of block * device was significant \((F(1.608,
1385.960) = 4.763, p < 0.05)\).

Using Helmhert contrasts, the differences between blocks become
non-significant after block 1 \((p > 0.05)\). This again shows that
there was learning involved in block 1.

To account for learning with the combo and check box, results from block
6 only will be used to calculate the performance measures of measurement
time, throughput and error rate. The results from block 6 alone would
give a good measure of performance.

\subsection{Movement time}
\label{sec-results-movement}

The results showed that the mouse had an overall movement time of
\unit[1.3]{s} for all targets compared to \unit[1.6]{s} for the touch
screen. Therefore we can conclude that a mouse is on average 15.2\%
faster than a touch screen overlay. This is interesting as
\citet{Sear-A-1991-IJMMS} found that the movement time between a mouse
and touch screen (monitor) was similar for rectangle targets larger than
\unit[2]{mm}. Therefore the nature of the two types of touch screen
(overlay and monitor) may affect the movement time associated with the
type of touch screen. It is also likely that due to the loss of accuracy
found with the overlay during testing, the touch screen monitor will
have a lower movement time compared to the touch screen overlay.

The movement times for each target showed that the text box has the
fastest movement time, followed by the button, the check box and then
the combo box. These results are illustrated in
Figure~\ref{fig-movement-time} and follow Fitts' Law in that the largest
target (the text box) had the fastest movement time.

\begin{figure}
	\includegraphics{movement-time}
	\caption{Movement time for each target type across both devices and
	all sizes.}
	\label{fig-movement-time}
\end{figure}

As expected, the combo box had the slowest movement time due to the
two-click behaviour involved in making a selection. The sizes of the
combo box were exactly the same as the text box but movement time was
119\% slower. Thus the extra movement of selecting an item from the drop
down list increases the movement time involved with the combo box
dramatically. As the distance to the list item is relatively short from
the main combo box area, the significant increase in movement time is
therefore most likely due to users making more errors.

A touch screen has similar movement time to a mouse for medium and large
sized targets. But for the small targets, the touch screen was 67\%
slower than the mouse. The only time where the touch screen was found to
be faster than the mouse was with the largest target type - the large
text box.

The movement time for the small check box with the touch screen was 69\%
slower than that of the mouse. The small check box was the smallest item
tested having a width of \unit[4]{mm} and height of \unit[4]{mm}. We can
conclude that the touch screen was not efficient for selecting targets
as small as \unit[4]{mm}. \citet{Sear-A-1991-IJMMS} showed a touch
screen has similar movement time to the mouse for targets as small as
\unit[2]{mm}. Although a touch screen monitor can be used with targets
as small as \unit[2]{mm}, a touch screen overlay should only be used for
targets with a size of greater than \unit[4]{mm}. The results from the
error rate analysis also support this

\subsection{Throughput}
\label{sec-results-throughput}

Throughput for the mouse was \unit[1.238]{bps}, slightly higher than the
1.215 bps throughput for the touch screen. The device by itself was
shown not to have a significant effect on throughput \((F(1, 22) = 0.02,
p > 0.05)\). Throughput did not vary for size but throughput did vary
depending on target type \((F(2.07, 45.55) = 4.77, p < 0.001)\). Check
boxes had the highest throughput rate of \unit[1.967]{bps} (sd = 0.720).
This is interesting as the check box was shown to have the second worst
movement time and the worst error rate (see
Figure~\ref{fig-throughput}).

\begin{figure}
	\includegraphics{throughput}
	\caption{Throughput for each target type across both devices and all
	sizes.}
	\label{fig-throughput}
\end{figure}

Upon further investigation it was seen that the movement time for the
check box was in fact in the middle range of all targets and due to its
small size it had a high index of difficulty. Therefore these two
factors are the reason for the check box having such a high throughput
rate. The combo box had the worst throughput of \unit[0.501]{bps} (sd =
0.213). The index of difficulty was not very high for the combo box and
so it was due to its high movement time that the combo box had such a
low throughput rate.

The overall throughput rate of \unit[1.2]{bps} for the mouse is much
lower compared with previous research. A study by
\citet{Doug-SA-1994-SIGCHI} showed a mouse had a throughput rate of
\unit[4.15]{bps}. \citet{Mack-IS-1991} compared three devices (mouse,
tablet and trackball) using four target sizes (8, 16, 32 and 64 pixels)
over two different types of tasks: pointing and dragging. The throughput
for the mouse in this case was 4.5 bps. This may indicate the level of
difficulty with selection within this experiment is a lot higher than
within previous research. This could be due to the selection of GUI
targets instead of arbitrary rectangle targets.

\subsection{Error rate}
\label{sec-results-errorrate}

The error rate for the mouse was only 2.7\% which is consistent with
previous studies. The touch screen on the other hand had an error rate
of 60.7\%. \citet{Sear-A-1991-IJMMS} found that the touch screen had
an average error rate of 49\% but this was across much smaller targets.
This suggests there is a loss in accuracy from using a touch screen
overlay compared to a touch screen monitor.

The check box had a significantly high amount of errors; 78.5\% for all
sizes and both devices and in particular, 312.5\% for the small check box
with the touch screen. A 100\% error rate indicates one wrong selection
made for every correct selection. The touch screen incurred the majority
of the errors. With the check box the mouse had an error rate of 4.4\%
and the touch screen had an error rate 152.5\%. The distinguishing factor
of the check box compared to the other targets was its small size. We
can conclude from this that the touch screen overlay has inaccuracy in
selecting small targets (\unit[4]{mm} or less).

Buttons and text boxes had much lower error rates compared to that of
the check box and combo box (as seen in Figure~\ref{fig-error-rate}). As
buttons and text boxes also had low movement times, we can conclude that
these two targets have very good overall performance.

\begin{figure}
	\includegraphics{error-rate}
	\caption{Error rate for each target type across both devices and all
	sizes.}
	\label{fig-error-rate}
\end{figure}


\subsection{Comfort}
\label{sec-results-comfort}

In terms of accurate pointing the mouse (2.083) was rated easier than
the touch screen (3.000). These differences were statistically
significant \((p < 0.01)\). The responses regarding the question on
neck, wrist and arm fatigue showed that the touch screen had a high
rating (4.083), whereas the mouse was rated in the midpoint range
(3.167). These differences were statistically significant \((p < 0.5)\).
The final question rated the overall difficulty in using the selection
device. The mouse (4.250) was rated easier to use than the touch screen
(3.333). These differences were statistically significant \((p <
0.05)\).

For user satisfaction with the touch screen, both the text box and
button were rated easy to accurately select and the preferred size was
both the large size and the medium size. This feedback is consistent
with the data collected in that text boxes and buttons have short
movement time and low error rates (easy to accurately select).

The combo box was rated in the midpoint range in terms of ease in
accurately selecting with the touch screen and the check box was rated
very hard to select. Three quarters of the touch screen users preferred
to select the large size combo boxes and check boxes and this reflects
the poor error rates and movement times associated with these two
targets with the touch screen overlay.

The participants using the mouse rated the text box and button easy to
accurately select with the medium and large sizes being the most
preferred. Both the combo box and check box were rated harder to select
than the button and text box with the check box having the worst rating.
Like the touch screen overlay, the preferred size for the combo box and
check box was large.

One participant noted the lack of arm support for targets at the top of
the screen. This is an interesting comment because the nature of using a
touch screen means the users arm might be raised off the desk and be
self supporting when selecting items towards the top of the desktop
screen.

Another suggestion was making the target change colour when the cursor
is located above it. This is a similar concept to that of interactive
rollover items commonly used in web pages. Auditory feedback has been
shown to affect the speed and accuracy when making a selection
\citep{Bend-G-1999-PhD}, and so it likely the visual feedback received
from GUI targets will affect the selection performance. All the targets
being tested provide some form of immediate visual feedback from the
button being visually depressed it to a tick appearing in the check box.
Future study is needed to assess how visual feedback affects selection
performance and what the most effective method of providing feedback is.

\section{Conclusions}
\label{sec-conclusions}

The goal of the study was to assess the ability of a touch screen
overlay in selecting different targets commonly presented to users in an
information system. The touch screen overlay sits over a normal monitor
and results in a gap between the overlay and monitor itself. This gap
was shown to decrease the accuracy of the touch screen.

The results showed that the touch screen overlay was both slower and
less accurate than the mouse. The touch screen was found to have
reasonable performance with large GUI items but poor performance with a
smaller GUI items. The touch screen overlay did have comparable movement
times to the mouse for medium and large sized targets. Throughput did
not vary across device or size but did vary across target. Both
selection devices had the same user preference except with the smallest
target, check boxes, in which the mouse had a higher preference. The
mouse was rated easier to make accurate selections with than the touch
screen. The touch screen overlay also has worse arm, wrist and finger
fatigue compared to the mouse. From these results we can conclude that
the mouse had higher user satisfaction than a touch screen.

In general we can conclude that a touch screen overlay with no external
device (i.e. pen) is not an effective selection device for targets
having a size of \unit[4]{mm} or smaller. When designing interfaces that
will be used with a touch screen overlay, selection within the interface
will be more efficient if the GUI items are larger than \unit[4]{mm}.

Although the results showed that the touch screen overlay was not
efficient and usable for selecting items with a size of \unit[4]{mm} or
less, this may not be the case when a pen or some external device is
used in conjunction with the touch screen overlay. In general there
seems to a lack of research done in device assessment with touch screens
and pens or other external devices. Further testing on touch screens
used with an external device such as a pen may well show that a touch
screen overlay is adequate and efficient for selecting small items
(\unit[4]{mm} or less).


% The Appendices part is started with the command \appendix;
% appendix sections are then done as normal sections
% \appendix

% \section{}
% \label{}

% Bibliographic references with the natbib package:
% Parenthetical: \citep{Bai92} produces (Bailyn 1992).
% Textual: \citet{Bai95} produces Bailyn et al. (1995).
% An affix and part of a reference:
%   \citep[e.g.][Ch. 2]{Bar76}
%   produces (e.g. Barnes et al. 1976, Ch. 2).

\bibliographystyle{elsart-harv}
\bibliography{IwC_paper}
% \begin{thebibliography}{}

% \bibitem[Names(Year)]{label} or \bibitem[Names(Year)Long names]{label}.
% (\harvarditem{Name}{Year}{label} is also supported.)
% Text of bibliographic item

% \bibitem[]{}
% 
% \end{thebibliography}

\end{document}

